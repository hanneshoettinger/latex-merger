%\documentclass[times]{MGS_class}
%-----
%Title und Author angeben
%\title{This is a test article\\
%for the Master\\
%Game Engineering and Simulation}
%\author{NAME \textsuperscript{1} und NAME \textsuperscript{1}}
%%die Institute weiter unten zu den zugehörigen superscripts angeben
%
%%\begin{document}
%
%\twocolumn[{\csname @twocolumnfalse\endcsname
%\begin{center}
%\maketitle
%%\thispagestyle{empty}
%
%%------
%%ANGABE DER INSTITUTE ANFANG
%
%\noindent \scriptsize \textsuperscript{1}\textsl{FH Technikum Wien, Game Engineering und Simulation, Wien, AUT}\\
%
%%ANGABE DER INSTITUTE ENDE
%%------
%\end{center}
%%\normalsize
%\hspace{15mm}
%
%}]%---

\begin{abstract}
2nd File
\end{abstract}
\begin{keywords}
algorithm $\beta$-phase testing, group-velocity matching, AI
\end{keywords}

\section{Einleitung} 

Link (\url{www.seniam.org}).

Code:

\begin{lstlisting}[label={list:first},caption=Sample code.]
#include <iostream>
 
int main()
{
  std::cout << "Hello World!" << std::endl;
  return 0;
}
\end{lstlisting}

Citation: \citep{lohse_how_2010}. \\
another Citation: \cite{smeets_throwing_2002} describe...\\
Bildreferenz \fref{subfig:label1} einleitet. 

Subfigure:

\begin{figure}[htbp] 
	\centering
	\subfigure[Caption1]
	{\includegraphics[width=0.7\columnwidth]{img/test1.png}
	\label{subfig:label1}}
	\subfigure[Caption2]
	{\includegraphics[width=0.7\columnwidth]{img/test2.png}
	\label{subfig:label2}}
	\caption{Subfigure Caption}
	\label{fig:figure1}
\end{figure} 

% this is a pagebreak
\pagebreak

\section{Methoden}


\begin{description}
\item \textbf{Description}
\begin{enumerate}
\item text text text
\item text text text
\item text text text
\end{enumerate}
\end{description}


\section{Ergebnisse}

Tabelle:

\begin{table}[htbp]
\centering
\begin{tabular}{c|c}
Bereich & f [Hz] \\ \hline \hline
Anfang & 194 \\ \hline
Mitte & 229 \\ \hline
Ende & 220
\end{tabular}
\caption{Ergebnis Median Frequenzen}
\label{tab:medfrequ}
\end{table}

\section{Diskussion}


Start here.
This is math in text: $\frac{\alpha + 2}{\sqrt{10 + x^2}}$.


({\it e.g.} [5, 9, 11, 12, 15, 16, 19])

Start equation here:

\begin{equation}
{\rm GVM}_{F,SH} = \left({\rm GV}_F \cos \theta \right )^{-1} -
\left ({\rm GV}_{SH}\right)^{-1} = 0
\end{equation}

or

\begin{equation}
\begin{array}{rcl}
\displaystyle{\theta_{\rm int} \left(\lambda_F\right)} & = &
\displaystyle{\arcsin \left[ n_F
\left(\lambda_{F}^* \right) \sin \theta^* / n_F \left(\lambda_F
\right)\right]}\\
& = & \displaystyle{\arcsin \big [\sin \theta_{\rm ext} / n_F
(\lambda_F)\big]~.}
\end{array}
\end{equation}

% this is a columnbreak
\vfill\break

Include a figure here:

\begin{figure}[!hbp]
\includegraphics[scale = 0.5]{img/test3.png}
\caption{ caption \protect with citation \cite{smeets_throwing_2002}}
\label{fig:test1}
\end{figure}

in \fref{fig:test1} - this was a reference to a figure using the label!

% Bibliography



% Option 1: Bibliography generation with BibTeX:
% Add all refernces to the *.bib file (here Literatur.bib). Then run LaTeX, BibTeX, and again twice LaTeX.
% Use \cite{kop05} within the text for a citation.
% - GEMRAN ----------
%\bibliographystyle{apalike-url-de} %ermöglicht longnamesfirst Option. Bei erster Erwähnung werden alle Autoren gennannt in der Folge et al. verwenden wir aber nicht
% - END GERMAN-------
%% -ENGLISH---------

\bibliographystyle{apalike-url-de}
%% -END ENGLISH------
\bibliography{Bibliography}
% Option 2: Bibliography generation without BibTeX:
%\begin{thebibliography}{99}
%\bibitem[kop05]{kop05}
%H.~Kopka, {\em LaTeX, Band 1: Einf"uhrung}, Pearson Studium, M"unchen, 3.~Auflage, 2005.
%\bibitem[knu98]{knu98}
%F.~Mittelbach, M.~Goossens, J.~Braams, D.~Carlisle, and Ch. Rowley, {\em The LaTeX Companion}, 
%Addison-Wesley, 2nd edition, 2004.
%\end{thebibliography}

%\end{document}